\documentclass[a4paper,fleqn]{cas-dc}
%\documentclass[a4paper,fleqn,longmktitle]{cas-dc}
\usepackage[numbers]{natbib}
%\usepackage[authoryear]{natbib}
%\usepackage[authoryear,longnamesfirst]{natbib}

% ==============================================================================
% 【核心基础】LaTeX基础排版 & 编码支持(必选)
% ==============================================================================
\usepackage[T1]{fontenc}          % 字体编码(支持特殊字符、连字,提升PDF字符显示质量)
%\usepackage[utf8]{inputenc}       % 输入编码(支持中文/英文混合输入,避免乱码)
\usepackage{lmodern}              % 拉丁现代字体(替代默认Computer Modern,更清晰,支持矢量缩放)
\usepackage{textcomp}             % 文本符号扩展(如度数°、版权©、欧元€、千分号‰等,适配学术符号)

% ==============================================================================
% 【智能引用】交叉引用增强(核心推荐)
% ==============================================================================
\usepackage{cleveref}             % 智能交叉引用(自动识别引用类型,替代\ref,支持\cref/\Cref)
% 定制\cref引用名称(适配Elsevier期刊命名习惯)
\crefname{figure}{Fig.}{Figs.}    % 图引用:单数Fig.,复数Figs.(符合Elsevier缩写规范)
\crefname{table}{Tab.}{Tabs.}     % 表引用:单数Tab.,复数Tabs.(替代全称Table,更简洁)
\crefname{table}{Table}{Tables}   % 可选:若期刊要求全称,启用此行
\crefname{algocf}{Algorithm}{Algorithms}  % algorithm2e环境引用名称
\Crefname{algocf}{Algorithm}{Algorithms}  % \Cref首字母大写版本(如开头引用)
\crefname{equation}{Eq.}{Eqs.}    % 公式引用:单数Eq.,复数Eqs.
\crefname{section}{Section}{Sections}      % 章节引用:单数Section,复数Sections
% 可选:伪代码行号引用(若用algpseudocode行号)
% \crefname{ALC@line}{line}{lines}
% \Crefname{ALC@line}{Line}{Lines}

% ==============================================================================
% 【算法伪代码】排版(二选一,优先algorithm2e)
% ==============================================================================
% 方案1:功能强的伪代码包(推荐,适配复杂算法)
\usepackage[ruled,vlined]{algorithm2e}  % 伪代码核心包
% [ruled]:顶部/底部横线;[vlined]:竖线分隔算法块;[algochapter]:算法按章节编号
\SetKwInput{KwInput}{Input}       % 自定义伪代码关键字:Input
\SetKwInput{KwOutput}{Output}     % 自定义伪代码关键字:Output
\SetAlFnt{\small}                 % 算法字体大小(适配Elsevier排版要求)
\SetNoFillComment                 % 注释不填充整行,更紧凑

% 方案2:传统伪代码包(备选,语法简洁)
% \usepackage{algorithm}           % 算法外层容器
% \usepackage{algpseudocode}       % 类Python伪代码语法

% ==============================================================================
% 【图表排版】核心 + 进阶(学术论文必备)
% ==============================================================================
\usepackage{graphicx}             % 插图核心包(支持PNG/JPG/PDF,\includegraphics,缩放/裁剪)
\graphicspath{{figs/}}         % 统一指定图片目录(避免重复写路径,建议新建figs文件夹)
\usepackage{float}                % 固定浮动体位置([H]强制图表在当前位置,避免乱浮动)
\usepackage{booktabs}             % 高质量三线表(\toprule/\midrule/\bottomrule,无竖线更规范)
\usepackage{longtable}            % 跨页长表格(解决表格内容超一页的问题)
\usepackage{subcaption}          % 子图表排版(\subfigure/\subtable,支持独立引用)
\usepackage{caption}              % 图表标题定制(统一样式,适配Elsevier)
\captionsetup{
	font=small,                   % 标题字体大小
	labelfont=bf,                 % 标签(如Fig.1)加粗
	skip=5pt,                     % 标题与图表间距
	justification=justified       % 标题左对齐(替代默认居中)
}
\usepackage[table]{xcolor}        % 颜色支持([table]扩展表格背景色)
\definecolor{lightgray}{gray}{0.9}% 浅灰色(表格高亮,避免过深)
\definecolor{bluegray}{rgb}{0.9,0.95,1} % 蓝灰色(重点行背景)

% ==============================================================================
% 【数学公式】增强(适配理工科/深度学习论文)
% ==============================================================================
\usepackage{amsmath}              % 核心数学包(多行公式、矩阵、分式、上下标、\align环境)
\usepackage{amssymb}              % 数学符号扩展(集合、希腊字母变体、特殊运算符)
\usepackage{amsfonts}             % 数学字体(黑板粗体\mathbb{N}/\mathbb{R},花体\mathcal{L})
\usepackage{mathtools}            % amsmath扩展(解决amsmath痛点:如\mathclap对齐、多行括号)
\usepackage{amsthm}               % 定理/定义/证明环境(适配学术论文定理排版)
% 定制定理环境(适配Elsevier风格)
\newtheorem{theorem}{Theorem}[section]  % 定理(按章节编号)
\newtheorem{definition}{Definition}[section]  % 定义

% ==============================================================================
% 【文本排版】进阶优化(提升可读性,适配期刊规范)
% ==============================================================================
\usepackage{microtype}            % 微排版优化(字符间距、字距调整,减少断行,提升PDF阅读体验)
\usepackage{enumitem}             % 列表定制(有序/无序列表,解决默认缩进混乱)
\setlist{
	leftmargin=2em,               % 列表缩进
	itemsep=0.3em,                % 列表项间距
	topsep=0.5em,                 % 列表与上文间距
	labelsep=0.5em                % 标签与内容间距
}
\usepackage{multicol}             % 多列排版(支持双栏/多栏,适配Elsevier双栏格式)
\usepackage{cuted}                % 双栏文档中插入通栏内容(如通栏图表、摘要)
\usepackage{seqsplit}             % 长文本自动换行(长URL、长字符串、长公式,避免超出页面)
\usepackage{url}                  % URL/邮箱处理(自动换行,避免截断)
\usepackage{lipsum}               % 测试文本生成(\lipsum[1-3],撰写模板时填充占位)
\usepackage{ragged2e}             % 文本对齐优化(\justifying强制两端对齐,避免英文单词间距过大)
\usepackage{hyphenat}             % 连字符控制(解决长单词断行问题,\hyphenation{al-go-rithm})

% ==============================================================================
% 【超链接】PDF交互增强(必选,最后加载)
% ==============================================================================
\usepackage{hyperref}             % 超链接核心包(目录、引用、URL跳转)
\hypersetup{
	colorlinks=true,              % 链接着色(替代边框,更美观)
	linkcolor=blue,               % 内部引用(图表/章节)颜色
	citecolor=green,              % 参考文献引用颜色
	urlcolor=cyan,                % URL链接颜色
	pdfborder={0 0 0},            % 去除链接边框
	pdfencoding=unicode,          % 支持Unicode编码
	psdextra,                     % 扩展PDF功能
	bookmarks=true,               % 生成PDF书签(目录)
	bookmarksnumbered=true        % 书签显示章节编号
}

% ==============================================================================
% 【特殊需求】可选实用包(按需启用)
% ==============================================================================
\usepackage{siunitx}              % 单位排版(规范物理/工程单位:\SI{50}{\milli\meter},避免单位混乱)
% siunitx配置(适配国际单位制)
\sisetup{
	detect-all,                   % 自动检测字体
	separate-uncertainty=true,   % 分离误差值(如\SI{10.5 \pm 0.2}{\second})
	math-micro=\mu,               % 微单位用μ(替代µ)
	text-micro=µ
}
\usepackage{verbatim}             % 代码/原始文本环境(\begin{verbatim})
\usepackage{moreverb}             % verbatim扩展(支持行号、文件导入)
\usepackage{makecell}             % 表格单元格换行(\makecell{第一行\\第二行})
\usepackage{multirow}             % 表格跨行(\multirow{2}{*}{内容})
\usepackage{threeparttable}       % 表格注释(表格下方添加注记,适配学术表格规范)
\usepackage{rotating}             % 旋转图表/表格(\rotatebox{90}{内容},解决宽表格排版)
\usepackage{adjustbox}            % 图表缩放/裁剪(\adjustbox{width=0.8\textwidth}{图},更灵活)
\usepackage{ifpdf}                % 条件编译(区分PDF/PS格式,避免兼容性问题)

% ==============================================================================
% 【条件编译开关】用于生成不同版本的PDF
% ==============================================================================
% 使用说明:修改下面某一行的注释来选择要编译的版本(只需修改这一处!)
% \def\FULL{1}              % 完整版:包含图像摘要和高光点(投稿用正文)
% \def\WITHOUTABSTRACT{1}   % 无摘要版:不包含图像摘要和高光点
% \def\GRAPHICALONLY{1}     % 仅图像摘要(投稿系统用)
% \def\HIGHLIGHTSONLY{1}    % 仅高光点(投稿系统用)
\ifdefined\FULL\else
\ifdefined\WITHOUTABSTRACT\else
\ifdefined\GRAPHICALONLY\else
\ifdefined\HIGHLIGHTSONLY\else
% \def\FULL{1}                % 完整版(包含图像摘要和高光点)
\def\WITHOUTABSTRACT{1}      % 默认:正文版(无图像摘要/Highlights,日常写作更快)
\fi\fi\fi\fi

%%%Author macros
\def\tsc#1{\csdef{#1}{\textsc{\lowercase{#1}}\xspace}}
\tsc{WGM}
\tsc{QE}
%%%

% Uncomment and use as if needed
%\newtheorem{theorem}{Theorem}
%\newtheorem{lemma}[theorem]{Lemma}
%\newdefinition{rmk}{Remark}
%\newproof{pf}{Proof}
%\newproof{pot}{Proof of Theorem \ref{thm}}


\begin{document}
\let\WriteBookmarks\relax
\def\floatpagepagefraction{1}
\def\textpagefraction{.001}
%\let\printorcid\relax % 页面下方的ORCID(s),注释为显式

% Short title
% \shorttitle{<short title of the paper for running head>} 
\shorttitle{}    

% Short author
% \shortauthors{<short author list for running head>}
\shortauthors{San Zhang et al.}

% This is the title of the paper
\title[mode = title]{The Application and Challenges of Artificial Intelligence in Education} 
\tnotemark[1]

% 标题脚注(写资助项目)
\tnotetext[1]{This document is the results of the research project funded by the National Science Foundation}

%作者信息
\author[1,2]{San Zhang}[
%       type=editor,
%       style=chinese,
%       auid=000,
%       bioid=1,
       prefix=Dr.,
%       role=Researcher,
       orcid=0000-0000-0000-yyyy,
%       facebook=<facebook id>,
%       twitter=<twitter id>,
%       linkedin=<linkedin id>,
%       gplus=<gplus id>
]

% Email id of the first author
\ead{szhangemai@email.com}
% Corresponding author indication
\cormark[1]

% Footnote of the first author
%\fnmark[1]

% URL of the first author
%\ead[url]{}

% Credit authorship
% eg: \credit{Conceptualization of this study, Methodology, Software}
\credit{Conceptualization of this study, Methodology, Software, Writing- Reviewing and Editing}

\author[2]{Si Li}[
prefix=Mr.,
% role=Researcher,
orcid=0000-0000-0000-xxxx]
\ead{sliemai@email.com}
\credit{Data curation, Writing- Original draft preparation.}

% 第一单位
\address[1]{organizatio1, addressline1, city1 postcode1, state1, country1} 
% 第二单位
\address[2]{organizatio2, addressline2, city2 postcode2, state2, country2} 

% 通讯作者脚注
\cortext[1]{Corresponding author}

% Abstract
% Here goes the abstract
\begin{abstract}
This article explores the current applications, potential benefits, and ethical and practical challenges of artificial intelligence (AI) in modern education. It first examines specific use cases of AI in instructional support and learning assessment; then discusses its impact on educational equity; and finally identifies key barriers to widespread adoption and suggests future directions. Research indicates that when appropriately implemented, AI can significantly enhance teaching efficiency and personalization, but its deployment must be accompanied by supportive policies and ethical guidelines to ensure fairness and integrity in education.

\end{abstract}

% 根据编译标志选择输出内容
% ============================================================================
% 仅图像摘要:生成独立PDF(不含正文)
\ifdefined\GRAPHICALONLY
\begin{graphicalabstract}
% 写作期加速(可选):如果图像摘要还没最终定稿/图片很大,启用 draft 仅显示占位框,可显著减少编译时长
% 用法:把下面这一行改成带 draft 的形式;定稿后再去掉 draft 即可
% \includegraphics[draft,width=0.7\linewidth]{figs/ga-figure.pdf}
\includegraphics[width=0.7\linewidth]{figs/ga-figure.pdf} 
\end{graphicalabstract}
\end{document}
\fi

% 仅高光点:生成独立PDF(不含正文)
\ifdefined\HIGHLIGHTSONLY
\begin{highlights}
\item highlight-1
\item highlight-2
\item highlight-3
\end{highlights}
\end{document}
\fi

% 完整版:包含图像摘要和高光点
\ifdefined\FULL
\begin{graphicalabstract}
% 写作期加速(可选):如果图像摘要还没最终定稿/图片很大,启用 draft 仅显示占位框,可显著减少编译时长
% 用法:把下面这一行改成带 draft 的形式;定稿后再去掉 draft 即可
% \includegraphics[draft,width=0.7\linewidth]{figs/ga-figure.pdf}
\includegraphics[width=0.7\linewidth]{figs/ga-figure.pdf} 
\end{graphicalabstract}

\begin{highlights}
\item highlight-1
\item highlight-2
\item highlight-3
\end{highlights}
\fi
% ============================================================================

% Keywords
\begin{keywords}
Artificial Intelligence (AI) \sep 
Educational Equity \sep 
Personalized Learning
\end{keywords}

\maketitle

% Main text(所有版本都包含正文)
\section{Specific Applications of AI in Teaching}
\subsection{Instructional Support Tools}
\subsubsection{Intelligent Lesson-Planning Systems}

Intelligent lesson-planning systems can automatically generate lesson plans, teaching slides, and practice exercises based on curriculum standards and student performance data. For instance, some platforms use natural language processing to analyze textbook content and recommend tailored instructional resources, substantially reducing teachers’ repetitive workload \cite{MOUNDRIDOU2024100277}.

\cref{tab:lesson_planning_metrics} presents the average lesson preparation time saved for teachers, the accuracy rate of resource recommendation, and the user satisfaction degree of an intelligent lesson planning platform across different disciplines.

\begin{table*}[ht]
	\centering
	\caption{Lesson-Planning System Performance Metrics}
	\label{tab:lesson_planning_metrics}
	\begin{tabular}{lccc}
		\hline
		\textbf{Subject} & \textbf{Time Saved (min/lesson)} & \textbf{Rec. Accuracy (\%)} & \textbf{Satisfaction (\%)} \\
		\hline
		Mathematics      & 22.5 & 89.3 & 92.1 \\
		Science          & 20.8 & 86.7 & 89.5 \\
		History          & 16.9 & 81.5 & 85.0 \\
		English          & 18.3 & 84.2 & 87.6 \\
		Foreign Language & 19.1 & 83.8 & 88.2 \\
		\hline
	\end{tabular}
\end{table*}

\subsubsection{Virtual Teaching Assistants and Chatbots}
Powered by large language models, virtual teaching assistants can answer students’ common questions and provide instant feedback around the clock. In higher education, such tools are already used for homework support in subjects like programming and mathematics, effectively alleviating pressure on teaching staff.

\subsection{Learning Assessment and Feedback}
\subsubsection{Automated Grading}
AI systems can automatically grade multiple-choice, fill-in-the-blank, and even short-answer questions. Advanced platforms can also evaluate essays for logical structure, grammatical accuracy, and writing style, offering multi-dimensional feedback that improves assessment efficiency \cite{10958994}.

As illustrated in \cref{fig:simple-png}, the scene depicts a digital classroom where a teacher and several students gather around a large screen displaying an automated grading system. The screen presents charts and pie graphs that reflect students’ academic performance and progress. The teacher is interacting with the system, while the students are checking their own score reports on their respective devices.

\begin{figure}
	\centering
	\includegraphics[width=0.7\linewidth]{figs/simple-png}
	\caption{Automated Grading}
	\label{fig:simple-png}
\end{figure}


\subsubsection{Learning Behavior Analytics}
By tracking metrics such as online study duration, clickstream patterns, and error distributions, AI constructs individualized learner profiles. These profiles help predict academic risks and deliver personalized intervention recommendations, enabling a data-driven realization of “teaching according to individual aptitude.”


\section{The Potential and Limits of AI in Promoting Educational Equity}

AI holds promise for narrowing educational disparities between urban and rural areas or across regions—for example, by delivering high-quality courses via online platforms to underserved communities. However, its real-world impact is constrained by uneven access to internet infrastructure, digital devices, and varying levels of digital literacy among teachers and students. Without adequate support, AI may inadvertently widen the “digital divide,” further marginalizing disadvantaged learners.

\section{Key Challenges in Scaling AI in Education}

The large-scale integration of AI into education currently faces three major obstacles:

\begin{enumerate}
	\item Data privacy and security concerns, as the collection and use of student behavioral data lack consistent regulatory frameworks;
	\item Algorithmic bias, which may lead to unfair assessments—particularly for students from non-dominant cultural or linguistic backgrounds; and
	\item Low teacher trust and proficiency in using AI tools, necessitating systematic professional development and institutional incentives.
\end{enumerate}

Moving forward, stakeholders should foster a human-centered AI education ecosystem that positions technology as a complement—not a replacement—to the irreplaceable human elements of teaching: empathy, mentorship, and moral guidance.

\section{Modeling Time Savings from AI Lesson Planning}

The time saved by a teacher using an intelligent lesson-planning system can be approximated as proportional to both the baseline preparation time and the system’s recommendation accuracy. Specifically, let $T_{\text{manual}}$ be the manual preparation time (in minutes) and $A$ the accuracy of resource recommendations, where $0 \leq A \leq 1$. Then the time saved is given by:
\begin{equation}
	\Delta T = \alpha \, T_{\text{manual}} \, A,
	\label{eq:time_saved}
\end{equation}
where $\alpha$ is an empirical efficiency factor satisfying $0 < \alpha < 1$. As shown in \cref{eq:time_saved}, higher accuracy directly translates to greater time savings.

\section{Conclusion}
Artificial intelligence holds significant promise for enhancing teaching efficiency and personalizing learning through tools like intelligent lesson planning and automated grading; however, its benefits are tempered by risks such as algorithmic bias, data privacy concerns, and the potential to widen educational inequities. To ensure AI serves as a force for inclusive and high-quality education, its deployment must be guided by human-centered design, robust ethical safeguards, and strong collaboration among educators, developers, and policymakers.

% To print the credit authorship contribution details
\printcredits
% https://www.elsevier.com/zh-cn/researcher/author/policies-and-guidelines/credit-author-statement

%% Loading bibliography style file
%\bibliographystyle{model1-num-names}
%\bibliographystyle{cas-model2-names}
\bibliographystyle{elsarticle-num}

% Loading bibliography database
\bibliography{cas-refs}

\end{document}
